\documentclass[french,a4paper]{book}
\usepackage{fontspec}

%opening
\title{Compte-rendu à t=2.0y}
\author{Jean-Michaël Celerier}

\begin{document}

\maketitle

\tableofcontents

\chapter{Introduction}
Ce document survole les travaux qui ont été réalisés jusqu'à présent 
lors de la thèse, étudie les pistes qui sont ouvertes et les possibilités 
pour la dernière année.

\section{Mise en relation avec le sujet}
Calques audio interactifs : théorie, mise en oeuvre et usages.

\section{Articulation et analyse générale}
\chapter{Réalisations}

\section{Développements théoriques, publications}
\subsection{États de l'art}

\subsection{Modèle théorique}
\subsubsection{TENOR2015: OSSIA}
\subsubsection{IUI2015 (refusé)}
\subsubsection{JNMR: Vérification}
\subsubsection{JIM2016: Interface}
\subsubsection{ICMC2016: Programmation structurée}

\subsection{Espace}
\subsubsection{JIM2016: Démo}
% TODO la mettre sur scholar
\subsubsection{JIM2016: Espace}
-> Conclusion : CAS peu adéquat, dur d'avoir de bonnes performances à un tick rate quelconque.
Alternatives : se restreindre aux cas linéaires ? 
GPU ? Mais latence.


\subsubsection{Compte-rendu espace}

\subsection{Audio}
\subsubsection{SMC2016: i-score et LibAudioStream}

\subsection{Répartition}
\subsubsection{Rapport de stage}

\section{Conférences, présentations, workshops}
\subsection{Cycles SCRIME 2015}
\subsection{Cycles SCRIME 2016}
\subsection{GDR ESARS}gitter.im
\subsection{DESINC2016}
\subsection{Forum IRCAM 2015}
\subsection{Workshop improvisation}

\section{Développements logiciels}

\subsection{Génie logiciel et généralités}
\subsubsection{Performances}
Question des performances ? Comment mettre en valeur ? 
Un accent très fort est mis dessus.

\subsubsection{Tests}
Idem pour tests.
Couverture de code : 70 \% pour libossia, 50 \% pour i-score, 0 \% pour extensions i-score

\subsection{i-score}
\subsubsection{Architecture}
\subsubsection{Problèmes actuels}
\subsubsection{Portabilité}
Après avoir enlevé Jamoma, exécution sur Android et iOS.
    
\subsection{Extensions à i-score}
\subsubsection{Édition répartie}
\subsubsection{Audio}
\subsubsection{Automation 3D}
\subsubsection{PureData}
\subsubsection{Espace}
\subsubsection{Image}
\subsubsection{Vidéo}
\subsubsection{Controle à distance}
\subsubsection{Analyse statique}
\subsubsection{Segments}
\subsubsection{Extension "Preset"}

\subsection{libossia}
\subsubsection{Architecture}
\subsubsection{Problèmes actuels}
- temps de compilation
\subsubsection{Portages}
\paragraph{C}
\paragraph{Csharp et Unity}
\paragraph{Qt}
\paragraph{Java}
\paragraph{Javascript}
\subsection{OSCQuery}
\subsection{coppa}

\subsection{Études et développements mineurs}
\subsubsection{External RealSense}
\subsubsection{Outils pour graphe de calcul}
\paragraph{DisPATCH}
\paragraph{RaftLib}

\subsubsection{Contribution à d'autres projets open-source}
\paragraph{LibAudioStream}
\paragraph{FAUST}
\paragraph{Jamoma}
\paragraph{Contributions mineures}
\begin{itemize}
\item Placeholder/Nodeeditor
\item verdigris
\item fmt
\item Qt-color-widgets
\item jni.hpp
\item quazip
\item QRecentFilesMenu
\item ModernMIDI
\item libsamplerate
\item ofxMSAPhysics
\item Cotire
\end{itemize}

\section{Projets liés}

\subsection{Audio}

\subsubsection{Stage Magali Chauvat}
\paragraph{Objectifs}

\subsection{Robots}
\subsubsection{Stage Nicolas 2015}
\subsubsection{Stage Kinda Al Chahid 2015}

\subsubsection{Stage Paul Breton 2016}
\subsubsection{Stage Maëva 2016}

\subsubsection{Projet TM - Robot 2015 - 2016}
\paragraph{Objectifs}
\paragraph{Groupe TM}
\paragraph{Groupe Robots}

\subsubsection{Projet TM - Robot 2016 - 2017}
\paragraph{Objectifs}
\paragraph{Groupe TM}
\paragraph{Groupe Robots}

\subsubsection{PFA2016 - 2017}
\paragraph{Objectifs}

\section{Cours et TDs donnés}
\subsection{TIM}
\subsection{TAP}

\chapter{Objectifs à venir}
\section{Système réparti}
\subsection{Exécution répartie}
\subsection{Répartition des protocoles}

\section{Audio}
\subsection{Article dans CMJ ?}
Pour que ce soit convainquant : offrir en plus la possibilité 
de réutiliser les flux passés. Et bien tout modéliser.

\subsection{Signatures temporelles}
\subsection{Support audio étendu}
\subsubsection{VST / VSTi}
\subsubsection{LV2}
Format de plug-ins qui permet l'analyse en temps réel de données.

\section{Embedding de i-score}
\subsection{DLL dans d'autres moteurs d'exécution}
\subsection{Scénarios compilés}
\subsection{Web}
\subsection{IncludeOS pour devices?}

\section{Unification temps - espace}
\section{Modèle par graphe de noeuds pour calcul par tranches}
\section{Objectifs personnels}
\chapter{Conclusion}

\end{document}
